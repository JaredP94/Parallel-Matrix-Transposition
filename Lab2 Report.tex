%%%%%%%%%%%%%%%%%%%%%%%%%%%%%%%%%%%%%%%%%%%%%%%%%%%%%%%%%%%%%%%%%%%%%%%%%%%%%%%
%
% witseiepaper-2005.tex
%
%                       Ken Nixon (12 October 2005)
%
%                       Sample Paper for ELEN417/455 2005
%
%%%%%%%%%%%%%%%%%%%%%%%%%%%%%%%%%%%%%%%%%%%%%%%%%%%%%%%%%%%%%%%%%%%%%%%%%%%%%%%%

\documentclass[10pt,twocolumn]{witseiepaper}
%
% All KJN's macros and goodies (some shameless borrowing from SPL)
\usepackage{KJN}
\usepackage[super]{nth}
\usepackage{subcaption}
\usepackage{listings}
\usepackage{amsmath,amsfonts,amssymb}
\usepackage{epstopdf}
\usepackage{xcolor}
\usepackage{textcomp}
\usepackage{listings}
\usepackage{alltt}
%\usepackage{matlab-prettifier}
\usepackage{graphicx}
\usepackage{changes}
\usepackage{makecell}
\usepackage{verbatim}
\usepackage{balance}
\usepackage{pdfpages}
\usepackage{ragged2e}
\usepackage{algorithm}
\usepackage{algorithmicx}
\usepackage{multirow}
\usepackage[noend]{algpseudocode}
\usepackage{color} %red, green, blue, yellow, cyan, magenta, black, white
\definecolor{mygreen}{RGB}{28,172,0} % color values Red, Green, Blue
\definecolor{mylilas}{RGB}{170,55,241}
%\usepackage{flafter}

%
% PDF Info
%
\ifpdf
\pdfinfo{
	/Title (INSTRUCTIONS AND STYLE GUIDELINES FOR THE PREPARATION OF FINAL YEAR LABORATORY PROJECT PAPERS : 2005 VERSION)
	/Author (Ken J Nixon)
	/CreationDate (D:200309251200)
	/ModDate (D:200510121530)
	/Subject (ELEN417/455 Paper Format, 2005)
	/Keywords (ELEN417, ELEN455, paper, instructions, style guidelines, laboratory project)
}
\fi

%%%%%%%%%%%%%%%%%%%%%%%%%%%%%%%%%%%%%%%%%%%%%%%%%%%%%%%%%%%%%%%%%%%%%%%%%%%%%%%
\begin{document}
	
	\begin{titlepage}
		
		\newcommand{\HRule}{\rule{\linewidth}{0.3mm}} % Defines a new command for the horizontal lines, change thickness here
		
		\center % Center everything on the page
		
		%----------------------------------------------------------------------------------------
		%	HEADING SECTIONS
		%----------------------------------------------------------------------------------------
		\includegraphics[width=0.3\textwidth]{EIE.png}\\[1cm] % Include a department/university logo - this will require the graphicx package
		
		%----------------------------------------------------------------------------------------
		\textsc{\LARGE University of the Witwatersrand } \\[0.1cm] % Name of your university/college
		\textsc{\LARGE School of Electrical and Information Engineering }\\[1cm] % Major heading such as course name
		\textsc{\Large ELEN4020: Data Intensive Computing}\\[1.5cm] % Minor heading such as course title
		
		%----------------------------------------------------------------------------------------
		%	TITLE SECTION
		%----------------------------------------------------------------------------------------
		
		\HRule \\[0.4cm]
		{ \huge \bfseries Laboratory Exercise 2} \\[0.4cm] % Title of your document
		\HRule \\[1.5cm]
		
		%----------------------------------------------------------------------------------------
		%	AUTHOR SECTION
		%----------------------------------------------------------------------------------------
		\textsc{\Large 	\emph{Authors:} } \\[0.1cm]	 
		
		
		\begin{minipage}{0.4\textwidth}
			\begin{flushleft} \large
				%			\emph{Author:} \\
				Kayla-Jade Butkow \\ 714227 % Your name
			\end{flushleft}
		\end{minipage}
		~
		\begin{minipage}{0.4\textwidth}
			\begin{flushright} \large
				%	\emph{Author:}\\
				Jared Ping \\ 704447
			\end{flushright}
		\end{minipage}\\[1cm]
		
		\begin{minipage}{0.4\textwidth}
			\begin{flushleft} \large
				%		\emph{Author:}\\
				Lara Timm \\ 704157
			\end{flushleft}
		\end{minipage}
		~
		\begin{minipage}{0.4\textwidth}
			\begin{flushright} \large
				%		\emph{Author:} \\
				Matthew van Rooyen \\ 706692
			\end{flushright}
		\end{minipage}\\[1cm]
		
		
		
		{\large Date Handed In: \nth{9} March, 2018}\\[1cm] 
		
	\end{titlepage}


\pagestyle{plain}
\setcounter{page}{1}
\onecolumn
%%%%%%%%%%%%%%%%%%%%%%%%%%%%%%%%%%%%%%%%%%%%%%%%%%%%%%%%%%%%%%%%%%%%%%%%%%%%%%%
%
\section{Matrix Transposition}
In order to allow for simple matrix transposition, the two-dimensional matrix is created as a one-dimensional array, populated in row-major form. The matrix is sequentially filled with integers corresponding to the index+1.

Transposition of the matrix begins with allocating memory for an array to store the indices of swapped values. For each existing index, a new index is calculated using Equation~\ref{newIndex} where N = number of rows = number of columns

\begin{equation}
\label{newIndex}
newIndex = (oldIndex*N)\%(N^2-1)
\end{equation}

\section{OpenMP}
The multi-threaded algorithm was implemented using OpenMP. Multi-threading was performed on all aspects of the process, including the populating of the matrix and the transposition of the array. 

For all of the processes using OpenMP, the loops were divided into chunks of the size of the matrix divided by 256. Each function takes in a parameter called \texttt{noOfThreads}, which allows the user to specify the number of threads that should be used to execute the process.

Since the matrix is large, populating it in serial takes a large amount of time. It was thus implemented in parallel. Since each loop takes a consistent amount of time (as the same process is performed in each loop), static scheduling was used \cite{HPC}. 

The transposition was also performed using static scheduling, as it proved to give the best performance. 

\section{PThreads}

\section{Comparison of Performance}

\begin{table}[h]
	\centering
	\caption{Performance of the algorithm when run as a serial process}
	\begin{tabular}{|c|c|c|}
	\hline
	  N$_{0}$ = N$_{1}$ = 128 &  N$_{0}$ = N$_{1}$ = 1024 & N$_{0}$ = N$_{1}$ = 8192 \\
		\hline 
		  &  &  \\ 
		\hline 
	\end{tabular} 
\end{table} 

\begin{table}[h]
\centering
\caption{Performance of the algorithm using 4 threads}
\begin{tabular}{|c|c|c|c|}
	\hline 
	 & N$_{0}$ = N$_{1}$ = 128 &  N$_{0}$ = N$_{1}$ = 1024 & N$_{0}$ = N$_{1}$ = 8192 \\ 
	\hline 
	PThread &  &  &  \\ 
	\hline 
	OpenMP &  &  &  \\ 
	\hline 
\end{tabular}
\end{table} 

\begin{table}[h]
		\centering
\caption{Performance of the algorithm using 8 threads}
\begin{tabular}{|c|c|c|c|}
	\hline 
	 & N$_{0}$ = N$_{1}$ = 128 &  N$_{0}$ = N$_{1}$ = 1024 & N$_{0}$ = N$_{1}$ = 8192 \\ 
	\hline 
	PThread &  &  &  \\ 
	\hline 
	OpenMP &  &  &  \\ 
	\hline 
\end{tabular} 
\end{table}

\begin{table}[h]
		\centering
	\caption{Performance of the algorithm using 16 threads}
\begin{tabular}{|c|c|c|c|}
	\hline 
	 & N$_{0}$ = N$_{1}$ = 128 &  N$_{0}$ = N$_{1}$ = 1024 & N$_{0}$ = N$_{1}$ = 8192 \\ 
	\hline 
	PThread &  &  &  \\ 
	\hline 
	OpenMP &  &  &  \\ 
	\hline 
\end{tabular} 
\end{table}

\begin{table}[h]
		\centering
	\caption{Performance of the algorithm using 64 threads}
\begin{tabular}{|c|c|c|c|}
	\hline 
	 & N$_{0}$ = N$_{1}$ = 128 &  N$_{0}$ = N$_{1}$ = 1024 & N$_{0}$ = N$_{1}$ = 8192 \\ 
	\hline 
	PThread &  &  &  \\ 
	\hline 
	OpenMP &  &  &  \\ 
	\hline 
\end{tabular} 
\end{table}

\begin{table}[h]
	\centering
	\caption{Performance of the algorithm using 128 threads}
	\begin{tabular}{|c|c|c|c|}
		\hline 
		& N$_{0}$ = N$_{1}$ = 128 &  N$_{0}$ = N$_{1}$ = 1024 & N$_{0}$ = N$_{1}$ = 8192 \\ 
		\hline 
		PThread &  &  &  \\ 
		\hline 
		OpenMP &  &  &  \\ 
		\hline 
	\end{tabular} 
\end{table}

%%%%%%%%%%%%%%%%%%%%%%%%%%%%%%%%%%%%%%%%%%%%%%%%%%%%%%%%%%%%%%%%%%%%%%%%%%%%%%%
%

\bibliographystyle{witseie}
\bibliography{individual}

%{\tiny \vfill \hfill \today \hspace{5mm} witseie-paper-2003.\TeX}

\newpage
\onecolumn
\pagenumbering{roman}
\setcounter{page}{1}
\begin{appendix} \label{sec:appendix}
		

\end{appendix}

\end{document}

" vim: ts=4
" vim: tw=78
" vim: autoindent
" vim: shiftwidth=4